\documentclass[12pt, letterpaper]{article}
%\usepackage{geometry}
\usepackage[inner=1.75cm,outer=1.75cm,top=1.75cm, bottom=1.75cm]{geometry}
\pagestyle{empty}
\usepackage{graphicx,multicol}
%\usepackage{pdfpages}
\usepackage{fancyhdr, lastpage, bbding, pmboxdraw}
\usepackage[usenames,dvipsnames]{color}
\definecolor{darkblue}{rgb}{0,0,.6}
\definecolor{darkred}{rgb}{.7,0,0}
\definecolor{darkgreen}{rgb}{0,.6,0}
\usepackage[colorlinks,pagebackref,pdfusetitle, urlcolor=darkblue,citecolor=darkblue, linkcolor=darkred,bookmarksnumbered,plainpages=false]{hyperref}
\renewcommand{\thefootnote}{\fnsymbol{footnote}}
\newcommand{\ddx}{\frac{d}{dx}}
\newcommand{\dydx}{\frac{dy}{dx}}
\newcommand{\ds}{\displaystyle}
\newcommand{\dy}{\frac{dy}{dx}}

\usepackage{tikzsymbols}

\usepackage{bchart}

\newcommand{\headervariable}{Chapter 3}

\pagestyle{fancyplain}
\fancyhf{}
\lhead{ \fancyplain{}{QUAN 2010, UCCS} }
%\chead{ \fancyplain{}{} }
\rhead{ \fancyplain{}{Course Notes:  \headervariable} }
%\rfoot{\fancyplain{}{page \thepage\ of \pageref{LastPage}}}
\fancyfoot[RO, LE]{\textbf{Chapter 3} page \thepage }
\thispagestyle{plain}

%%%%%%%%%%%% LISTING %%%
\usepackage{listings}
\usepackage{caption}
\DeclareCaptionFont{white}{\color{white}}
\DeclareCaptionFormat{listing}{\colorbox{gray}{\parbox{\textwidth}{#1#2#3}}}
\captionsetup[lstlisting]{format=listing,labelfont=white,textfont=white}
\usepackage{verbatim} % used to display code
\usepackage{fancyvrb}
\usepackage{acronym}
\usepackage{amsthm}
%\VerbatimFootnotes % Required, otherwise verbatim does not work in footnotes!

\usepackage{mathrsfs}


\usepackage{arydshln} %For dashed lines in tabular environments.
\usepackage{amssymb} %For \square.
\usepackage{amsmath} %For align* and other things.
\DeclareMathOperator{\csch}{csch}
\DeclareMathOperator{\sech}{sech}
\usepackage{enumerate}%,enumitem}



\usepackage{ulem} %For strikeout text.

\usepackage[final]{pdfpages} %For including PDF pages.

\usepackage{hyperref}

\newcommand{\laplace}{\mathscr{L}}
\newcommand{\su}{\mathcal{U}}



%\newcounter{LO}
%\newcounter{LOexample}
%\newcounter{LOtemp}

\newcounter{exercise}

\newcounter{visualconnection}

%\usepackage{tcolorbox} %For boxing the text.
\usepackage[skins]{tcolorbox}
\usepackage{pgf}

\newtcolorbox{statement}{colback=gray!10!white,colframe=black}

\newtcolorbox{exercise}{colback=white,colframe=green!50!black,fonttitle=\bfseries,colbacktitle=gray, title={\stepcounter{exercise} Exercise \theexercise}}

\newtcolorbox{contd}{colback=white,colframe=green!50!black,fonttitle=\bfseries,colbacktitle=gray, title={Exercise \theexercise, cont'd}}

\newtcolorbox{learninggoal}{skin=enhanced, colback=white, colframe=black, fonttitle=\bfseries, colbacktitle=gray!10, coltitle=green!50!black, attach boxed title to top left={xshift=-2mm,yshift=-2mm}, title={{\Large L}EARNING~~{\Large G}OAL}}

\newtcolorbox{defn}{skin=enhanced, colback=white, colframe=black, fonttitle=\bfseries, colbacktitle=gray!10, coltitle=green!50!black, attach boxed title to top left={xshift=-2mm,yshift=-2mm}, title={{\Large D}EFINITION}}

\newtcolorbox{theorem}{skin=enhanced, colback=white, colframe=black, fonttitle=\bfseries, colbacktitle=gray!10, coltitle=green!50!black, attach boxed title to top left={xshift=-2mm,yshift=-2mm}, title={{\Large T}HEOREM}}

\newtcolorbox{question}{skin=enhanced, colback=white, colframe=black, fonttitle=\bfseries, colbacktitle=gray!10, coltitle=green!50!black, attach boxed title to top left={xshift=-2mm,yshift=-2mm}, title={{\Large Q}UESTION}}


\newtcolorbox{warning}{skin=enhanced, colback=gray!10!white, colframe=black, fonttitle=\bfseries, colbacktitle=white, coltitle=red!50!gray, attach boxed title to top left={xshift=3mm,yshift=-2mm}, title={\large Warning!}}

\newtcolorbox{visualconnection}{skin=enhanced, colback=white, colframe=black, fonttitle=\bfseries, colbacktitle=white, coltitle=blue!50!gray, attach boxed title to top left={xshift=3mm,yshift=-2mm}, title={\large\stepcounter{visualconnection} Visual Connection \Alph{visualconnection}}}

\newtcolorbox{remark}{colback=white,colframe=black}



%\usepackage[latin1]{inputenc} %Needed for accented characters?
%\usepackage{amsfonts}
%\usepackage{latexsym}

%To print solutions, use \solutionstrue; To hide solutions, use \solutionsfalse.
%\sol takes two arguments. #1 is the vertical length. #2 is the text.


\newif\ifsolutions
\solutionsfalse

\ifsolutions
    \newcommand{\soln}[2]{\begin{minipage}[c][#1]{\linewidth}{\textcolor{blue}{\textbf{Solution:}}\quad \textcolor{blue}{#2}}\end{minipage}}
    \newcommand{\opsoln}[1]{#1}
    \newcommand{\tblsoln}[1]{\textcolor{blue}{#1}}
\else
    \newcommand{\soln}[2]{\begin{minipage}[c][#1]{\linewidth}{\vfill}\end{minipage}}
    \newcommand{\opsoln}[1]{0}
    \newcommand{\tblsoln}[1]{\textcolor{white}{#1}}
\fi

\ifsolutions
    \newcommand{\sol}[2]{\begin{minipage}[c][#1]{\linewidth}{\textcolor{blue}{}\quad \textcolor{blue}{#2}}\end{minipage}}
    \newcommand{\opsol}[1]{#1}
    \newcommand{\tblsol}[1]{\textcolor{blue}{#1}}
\else
    \newcommand{\sol}[2]{\begin{minipage}[c][#1]{\linewidth}{\vfill}\end{minipage}}
    \newcommand{\opsol}[1]{0}
    \newcommand{\tblsol}[1]{\textcolor{white}{#1}}
\fi


\renewcommand*\contentsname{Table of Contents}


\usepackage{tocloft}
\setlength\cftparskip{7pt}

%From IODE:
\newcommand{\vs}{\vskip.2cm} %customizable command for inserting small vertical space.  Usually appears between paragraphs.
\usepackage[inline,shortlabels]{enumitem} % gives ability to continue with numbering (add [resume] after \begin{enumerate}) AND to make horizontal lists by adding * to enumerate (\begin{enumerate*})

%\newcommand{\ds}{\displaystyle}
\newcommand{\vv}{\vec{v}}
\newcommand{\uu}{\vec{u}}
\newcommand{\yy}{\vec{y}}

\newcommand{\ww}{\textbf{w}}
\newcommand{\xx}{\textbf{x}}
\newcommand{\bb}{\textbf{b}}
\newcommand{\dt}{\frac{d}{dt}}

\newcommand{\RR}{\mathbb{R}}

\newtheorem{thm}{Theorem}
\newtheorem{ex}[thm]{Example}


\theoremstyle{definition}
%\newtheorem{defn}[thm]{Definition}

\begin{document}

%\setcounter{page}{1}
\pagenumbering{arabic}


\begin{center}

{\LARGE \textsc{Chapter 3:  Calculating Descriptive Statistics}}
\end{center}

\noindent In this chapter, we'll learn how to compute descriptive statistics like the mean and standard deviation both by hand and with Excel.

\section*{Measures of Central Tendency}

In statistics, ``average" or ``typical" values are known as \textbf{measures of central tendency}.

\begin{defn}
The \textbf{mean} ($\mu$ or $\overline{x}$) is the most common measure of central tendency and is calculated by summing all the data points and dividing by the size of the data set.
\begin{center}
\textbf{Excel formula:}~~~~~~~~~~~~~~
\end{center}
\end{defn}

\begin{tcolorbox}
The mean is the sum of the data items divided by the number of items.
$$ \text{Mean}=\frac{\sum x}{n}, $$
where $\sum x$ represents the sum of all the data items and $n$ represents the number of items.
\end{tcolorbox}

\noindent The mean of a sample is symbolized by $\bar{x}$, and the mean of an entire population is symbolized by $\mu$ (the lowercase Greek letter \textit{mu}).

\vskip.15in

\begin{exercise}  The table below shows the ten highest-earning TV actors and the ten highest TV actresses for the 2010-2011 television season.\\
\vskip.15in
\begin{tabular}{|c|c|c|c|}\hline
\textbf{Actor} & \textbf{Earnings} & \textbf{Actress} & \textbf{Earnings}\\
 & (millions of dollars) &  & (millions of dollars)\\ \hline
Charlie Sheen & $\$ 40$ & Eva Longoria & $\$13$\\ \hline
Ray Romano & $\$20$ & Tina Fey & $\$13$\\ \hline
Steve Carell & $\$15$ & Marcia Cross & $\$10$\\ \hline
Mark Harmon & $\$13$ & Mariska Hargitay & $\$10$\\ \hline
Jon Cryer & $\$11$ & Marg Helgenberger & $\$10$\\ \hline
Laurence Fishburne & $\$11$ & Teri Hatcher & $\$9$\\ \hline
Patrick Dempsey & $\$10$ & Felicity Huffman & $\$9$\\ \hline
Simon Baker & $\$9$ & Courteney Cox & $\$7$\\ \hline
Hugh Laurie & $\$9$ & Ellen Pompeo & $\$7$\\ \hline
Chris Meloni & $\$9$ & Julianna Margulies & $\$7$\\ \hline
\end{tabular}

\begin{enumerate}[(a)]
\item Find the mean earnings, in millions of dollars, for the ten highest-earning actors.

\vfill

\item Find the mean earnings in millions of dollars for the ten highest-earning actresses.

\vfill

\end{enumerate}
\end{exercise}

\vfill


\newpage

\begin{defn}
The \textbf{median} is the value in the data set for which half of the observations are higher and half are lower.
\begin{center}
\textbf{Excel formula:}~~~~~~~~~~~~~~
\end{center}
\end{defn}

\begin{tcolorbox}
\textbf{The Median}\\
To find the \textbf{median} of a group of data items,
\begin{enumerate}[(a)]
\item Arrange the data items in order, from smallest to largest.
\item If the number of data items is odd, the median is the data item in the middle of the list.
\item If the number of data items is even, the median is the mean of the two middle data items.
\end{enumerate}
\end{tcolorbox}


\begin{exercise}  Find the median for each of the following groups of data:
\begin{enumerate}[(a)]
\item $84,90,98,95,88$
\vfill
\item $68,74,7,13,15,25,28,59,34,47$
\vfill
\end{enumerate}
\end{exercise}

\vfill


\begin{exercise}
How are the mean and median related to each other in each of the following frequency distribution shapes?  Why?

\begin{enumerate}[(a)]

\item symmetric:
\vfill

\item left-skewed:
\vfill

\item right-skewed:
\vfill

\end{enumerate}

\end{exercise}

\vfill


\newpage


\begin{tcolorbox}
\textbf{Position of the Median}\\
If $n$ data items are arranged in order, from smallest to largest, the median is the value in the $\displaystyle \frac{n+1}{2}$ position.
\end{tcolorbox}

\begin{exercise}  Find the median for the following group of data items.
$$ 1,2,2,2,3,3,3,3,3,5,6,7,7,10,11,13,19,24,26 $$
\end{exercise}

\vfill


\begin{question}  Statisticians generally use the median instead of the mean when reporting income.  Why do you think this is?
\end{question}

\vskip.15in

\begin{exercise} Five employees in the assembly section of a television manufacturing company earn salaries of $\$ 19,700$, $\$20,400$, $\$ 21,500$, $\$22,600$, and $\$23,000$ annually.  The section manager has an annual salary of $\$95,000$.
\begin{enumerate}[(a)]
\item Find the median annual salary for the six people.
\vfill
\item Find the mean annual salary for the six people.
\vfill
\end{enumerate}
\end{exercise}

\vfill

\begin{defn}
An \textbf{outlier} is a value that is much higher or lower than most of the data.
\end{defn}


\newpage

\begin{defn}
The \textbf{mode} is the data value that occurs most often in a data set.  If more than one data value has the highest frequency, then each of these data values is a mode.  If there is no data value that occurs most often, then the data set has no mode.
\end{defn}

\begin{exercise}  Find the mode for each of the following groups of data:
\begin{enumerate}[(a)]
\item $7,2,4,7,8,10$
\vfill
\item $2,1,4,5,3$
\vfill
\item $3,3,4,5,6,6$
\vfill
\end{enumerate}
\end{exercise}

\vfill

\begin{exercise}  
Consider the following data values:  
$$ 5,13,18,2,16,18,5,20 $$
Calculate the mean, median, and mode by hand, and then describe the shape of the distribution.
\end{exercise}

\vfill

\newpage


\begin{exercise} \textbf{(Donnelly 3.6)}

The data in this lesson's Excel file lists monthly average values for the Dow Jones Industrial Index for some months form January 2016 until December 2017.
\begin{enumerate}[(a)]
\item Calculate the mean, median, and mode (using Excel).
\item Describe the shape of the distribution.
\end{enumerate}


\end{exercise}


\vspace{2in}


\section*{Measures of Variability}

\textbf{Measures of variability} (or \textbf{dispersion}) describe the spread of a data set.  We will define the most common ones.

\begin{defn}
The \textbf{range} is the difference between the highest and lowest data values in a data set.
$$ \text{Range}=\text{highest data value}-\text{lowest data value} $$
\end{defn}

\begin{exercise}

The figure below shows the age of the eight oldest U.S. presidents at the start of their first term.

\begin{bchart}[step=10,max=70]
\bcbar[text=Joseph Biden]{78}
\bcbar[text=Donald J. Trump]{70}
\bcbar[text=Ronald Reagan]{69}
\bcbar[text=William Henry Harrison]{68}
\bcbar[text=James Buchanan]{65}
\bcbar[text=George H. W. Bush]{64}
\bcbar[text=Zachary Taylor]{64}
\bcbar[text=Dwight D. Eisenhower]{62}
\end{bchart}


Find the age range for the eight oldest presidents.
\end{exercise}

\vfill




\begin{defn}
\textbf{Standard deviation} is another measure of variability, and standard deviation is found by determining how much each data item differs from the mean.

\textbf{Computing the Standard Deviation for a Data Set}
\begin{enumerate}
\item Find the mean of the data items.
\item Find the deviation of each data item from the mean:  $\text{data item}-\text{mean}$
\item Square each deviation:  $(\text{data item} - \text{mean})^2$
\item Sum the squared deviations:  $\sum(\text{data item} - \text{mean})^2$
\item Divide the sum in step 4 by $n-1$, where $n$ represents the number of data items:  $\displaystyle\frac{\sum (\text{data item} - \text{mean})^2}{n-1}$
\item Take the square root of the quotient in step 5.  This value is the standard deviation for the data set.
$$ \text{Standard deviation}=\sqrt{\frac{\sum (\text{data item} - \text{mean})^2}{n-1}}\text{  or  } s=\sqrt{\frac{\sum(x-\bar{x})^2}{n-1}} $$
\end{enumerate}
\end{defn} 


\begin{exercise}  Find the standard deviation for the ages of the seven presidents given above.
\end{exercise}


\vfill


\newpage


\begin{exercise} \textit{Interpreting Standard Deviation}\\
Shown below are the means and standard deviations of the yearly returns on two investments from 1926 through 2004.
\vskip.15in
\begin{tabular}{|c|c|c|}\hline
\textbf{Investment} & \textbf{Mean Yearly Return} & \textbf{Standard Deviation}\\ \hline
Small-Company Stocks & $17.5\%$ & $33.3\%$\\ \hline
Large-Company Stocks & $12.4\%$ & $20.4\%$\\ \hline
\end{tabular}

\begin{enumerate}[(a)]
\item Use the means to determine which investment provided the greater yearly return.
\vfill
\item Use the standard deviations to determine which investment had the greater risk.  Explain your answer.
\vfill
\end{enumerate}
\end{exercise}

\vfill
\vfill


\begin{defn}
The \textbf{variance} ($\sigma^2$ or $s^2$) measures the variability, or spread, of the data points in a set around the sets mean.
$$ s^2 = ~~~~~~~~~~~~~~ $$
\vspace*{.3in}
\begin{center}
In Excel: $~~~~~~~~~~~~~~$
\end{center}
\end{defn}


\begin{exercise} \textbf{(Donnelly, Your Turn 4)}

The data in this lesson's Excel file lists the number of books that seven adults have read during the last $12$ months.  
$$ 10,10,4,8,13,6,11 $$
Calculate the variance and standard deviation.

\end{exercise}

\vfill

\newpage

\section*{Using the Mean and Standard Deviation Together}

There are a few important concepts that use the mean and standard deviation together to describe data sets.


\begin{defn}
The \textbf{coefficient of variation (cv)} measures the standard deviation in terms of its percentage of the mean.
\begin{center}
formula:  ~~~~~~~~~~~~~~~~~~~~~~~~~~~~~~~~~~~~~~~~~~~
\end{center}
\vspace*{.2in}
\end{defn}

\begin{defn}
The \textbf{$z$-score} identifies the number of standard deviations a particular value is from the mean of its distribution.
\begin{center}
formula:  ~~~~~~~~~~~~~~~~~~~~~~~~~~~~~~~~~~~~~~~~~~~

\vspace*{.2in}

Excel formula:  ~~~~~~~~~~~~~~~~~~~~~~~~~~~~~~~~~~~~~~~~~~~
\end{center}
\vspace*{.2in}
\end{defn}

\begin{exercise}
The mean weight of newborn infants is $7$ pounds, and the standard deviation is $0.8$ pound.  The weights of newborn infants are normally distributed.  Find the $z$-score for a weight of 
\begin{enumerate}[(a)]
\item $9$ pounds
\item $7$ pounds
\item $6$ pounds
\end{enumerate}

\end{exercise}

\vfill

\begin{exercise}  

Intelligence quotients (IQs) on the Stanford-Binet intelligence test are normally distributed with a mean of $100$ and a standard deviation of $16$.  What is the IQ corresponding to a $z$-score of $-1.5$?

\end{exercise}

\vfill

\newpage


\begin{statement}

\textbf{The Empirical Rule} says that if a distribution follows a bell-shaped, symmetric curve centered around the mean, we should expect 
\begin{itemize}
\item approximately $68\%$ of the data to fall within one standard deviation of the mean,
\item approximately $95\%$ of the data to flal within two standard deviations of the mean, and
\item approximately $99.7\%$ of the data to fall within three standard deviations of the mean.
\end{itemize}

\begin{center}
\includegraphics[scale=.9]{Images/NormalDistnRule}

\textit{(Image from the book \textit{Thinking Mathematically} by Blitzer)}

\end{center}

\end{statement}

\begin{theorem}
\textbf{Chebyshev's Theorem:}  for any number $z$ greater than $1$, the percent of data points that fall within $z$ standard deviations from the mean is at least $\left( 1-\frac{1}{z^2}\right)\times 100$ for any distribution, regardless of its shape.
\end{theorem}


\begin{exercise}
Assume the average selling price for houses in a certain county is $\$348,000$ with a standard deviation of $\$30,000$.
\begin{enumerate}[(a)]
\item Determine the coefficient of variation.
\vfill
\item Calculate the $z$-score for a house that sells for $\$379,000$.
\vfill
\item Using the Empirical Rule, determine the range of prices that includes $\$68\%$ of the homes around the mean.
\vfill
\vfill
\item Using Chebyshev's Theorem, determine the range of prices that includes at least $91\%$ of the homes around the mean.
\vfill
\vfill
\vfill
\end{enumerate}
\end{exercise}


\newpage

\section*{Measures of Relative Position}

Measures of relative position compare the position of one value in relation to other values in the data set.

\begin{defn}
\textbf{Percentiles}\\
If $n\%$ of the items in a distribution are less than a particular data item, we say that the data item is in the \textbf{$n$th percentile} of the distribution.

\vspace*{.2in}
In Excel:
\vspace*{.2in}
\end{defn}



\begin{defn}
The \textbf{percentile rank} identifies the percentile of a particular value within a data set.

\vspace*{.2in}
In Excel:
\vspace*{.2in}
\end{defn}

\begin{exercise} A new baby is in the $50$th percentile for weight.  What does that mean?
\end{exercise}

\vfill


\begin{defn}
\textbf{Quartiles} are commonly encountered percentiles.  Quartiles divide data sets into four equal parts.
\begin{center}
\includegraphics[scale=.7]{Images/Quartiles}

\textit{(Image from the book \textit{Thinking Mathematically} by Blitzer)}
\end{center}

The first, second, and third quartiles in a data set are the values for the $25$th, $50$th, and $75$th percentiles, respectively.

\vspace*{.2in}
In Excel:
\vspace*{.2in}
\end{defn}


\begin{defn}
The \textbf{interquartile range (IQR)} describes the range of the middle $50\%$ of a data set.
$$ \text{Formula:}~~~~~~~~~~~~~~~~~~~~~~~~ $$

\vspace*{.3in}
\end{defn}

\begin{exercise} \textbf{(Donnelly, Your Turn 8)}

The data in this lesson's Excel file lists the U.S. and Canadian box-office revenues for the highest grossing films of all time (in millions of dollars).  Use Excel formulas to calculate the following:
\begin{enumerate}
\item The three quartiles and the IQR
\item The $65$th percentile
\item The percentile rank for the film \textit{Avatar}
\end{enumerate}

\end{exercise}

\vfill



\begin{defn}

\begin{itemize}

\item The list consisting of the minimum $Q1,Q2,Q3$, and maximum values of a data set is called \underline{~~~~~~~~~~~~~~~~~~~~~~~~~~~~~~~~~~~~~~~~}.

\vspace*{.3in}


\item The graphical display for this list is called the \underline{~~~~~~~~~~~~~~~~~~~~~~~~~~~~~~~~~~~~~~~~~~~~~~~~~~} and also includes any outliers.

\vspace*{.3in}

\end{itemize}

\end{defn}

\begin{exercise} \textbf{(Donnelly 3.76)}

The data in this lesson's Excel file indicates the battery life, in minutes, on a single charge, for $25$ iPads.

\begin{enumerate}[(a)]
\item Construct a box-and-whisker plot for the data.
\item Compute the five-number summary for the data.
\end{enumerate}

\end{exercise}

\vfill

\newpage

\section*{Measures of Association Between Two Variables}

So far, the statistics we've studied have all dealt with describing one variable at a time.  Measures of association describe the relationship between two variables.

\begin{defn}
\begin{itemize}
\item \textbf{sample covariance:}  measures the direction of the linear relationship between two variables.
\begin{itemize}
\item In Excel:
\end{itemize}
\item \textbf{sample correlation coefficient:}  measures \textbf{\underline{both}} the strength and direction of the linear relationship between two variables.
\begin{itemize}
\item In Excel:
\end{itemize}
\end{itemize}
\end{defn}

\begin{exercise} \textbf{(Donnelly 3.50)}

A regional manager at Acme markets would like to develop a model to predict weekly sales of pet food based on the shelf space.  The data in this lesson's Excel file shows the results collected from nine randomly selected stores.

\begin{enumerate}[(a)]
\item Calculate the sample covariance.
\item Calculate the sample correlation coefficient.
\item Describe the relationship between $x$ and $y$.
\end{enumerate}

\end{exercise}

\vfill

\noindent\hrulefill

\noindent\textbf{Examples:}

\begin{center}
\includegraphics[scale=.6]{Images/CorrelationCoefficients}

\textit{(Image from the book \textit{Thinking Mathematically} by Blitzer)}

\end{center}

\end{document}
