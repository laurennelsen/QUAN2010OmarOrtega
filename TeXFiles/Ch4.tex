\documentclass[12pt, letterpaper]{article}
%\usepackage{geometry}
\usepackage[inner=1.75cm,outer=1.75cm,top=1.75cm, bottom=1.75cm]{geometry}
\pagestyle{empty}
\usepackage{graphicx,multicol}
%\usepackage{pdfpages}
\usepackage{fancyhdr, lastpage, bbding, pmboxdraw}
\usepackage[usenames,dvipsnames]{color}
\definecolor{darkblue}{rgb}{0,0,.6}
\definecolor{darkred}{rgb}{.7,0,0}
\definecolor{darkgreen}{rgb}{0,.6,0}
\usepackage[colorlinks,pagebackref,pdfusetitle, urlcolor=darkblue,citecolor=darkblue, linkcolor=darkred,bookmarksnumbered,plainpages=false]{hyperref}
\renewcommand{\thefootnote}{\fnsymbol{footnote}}
\newcommand{\ddx}{\frac{d}{dx}}
\newcommand{\dydx}{\frac{dy}{dx}}
\newcommand{\ds}{\displaystyle}
\newcommand{\dy}{\frac{dy}{dx}}

\usepackage{tikzsymbols}

\usepackage{bchart}

\newcommand{\headervariable}{Chapter 4}

\pagestyle{fancyplain}
\fancyhf{}
\lhead{ \fancyplain{}{QUAN 2010, UCCS} }
%\chead{ \fancyplain{}{} }
\rhead{ \fancyplain{}{Course Notes:  \headervariable} }
%\rfoot{\fancyplain{}{page \thepage\ of \pageref{LastPage}}}
\fancyfoot[RO, LE]{\textbf{Chapter 4} page \thepage }
\thispagestyle{plain}

%%%%%%%%%%%% LISTING %%%
\usepackage{listings}
\usepackage{caption}
\DeclareCaptionFont{white}{\color{white}}
\DeclareCaptionFormat{listing}{\colorbox{gray}{\parbox{\textwidth}{#1#2#3}}}
\captionsetup[lstlisting]{format=listing,labelfont=white,textfont=white}
\usepackage{verbatim} % used to display code
\usepackage{fancyvrb}
\usepackage{acronym}
\usepackage{amsthm}
%\VerbatimFootnotes % Required, otherwise verbatim does not work in footnotes!

\usepackage{mathrsfs}


\usepackage{arydshln} %For dashed lines in tabular environments.
\usepackage{amssymb} %For \square.
\usepackage{amsmath} %For align* and other things.
\DeclareMathOperator{\csch}{csch}
\DeclareMathOperator{\sech}{sech}
\usepackage{enumerate}%,enumitem}



\usepackage{ulem} %For strikeout text.

\usepackage[final]{pdfpages} %For including PDF pages.

\usepackage{hyperref}

\newcommand{\laplace}{\mathscr{L}}
\newcommand{\su}{\mathcal{U}}



%\newcounter{LO}
%\newcounter{LOexample}
%\newcounter{LOtemp}

\newcounter{exercise}

\newcounter{visualconnection}

%\usepackage{tcolorbox} %For boxing the text.
\usepackage[skins]{tcolorbox}
\usepackage{pgf}

\newtcolorbox{statement}{colback=gray!10!white,colframe=black}

\newtcolorbox{exercise}{colback=white,colframe=green!50!black,fonttitle=\bfseries,colbacktitle=gray, title={\stepcounter{exercise} Exercise \theexercise}}

\newtcolorbox{contd}{colback=white,colframe=green!50!black,fonttitle=\bfseries,colbacktitle=gray, title={Exercise \theexercise, cont'd}}

\newtcolorbox{learninggoal}{skin=enhanced, colback=white, colframe=black, fonttitle=\bfseries, colbacktitle=gray!10, coltitle=green!50!black, attach boxed title to top left={xshift=-2mm,yshift=-2mm}, title={{\Large L}EARNING~~{\Large G}OAL}}

\newtcolorbox{defn}{skin=enhanced, colback=white, colframe=black, fonttitle=\bfseries, colbacktitle=gray!10, coltitle=green!50!black, attach boxed title to top left={xshift=-2mm,yshift=-2mm}, title={{\Large D}EFINITION}}

\newtcolorbox{theorem}{skin=enhanced, colback=white, colframe=black, fonttitle=\bfseries, colbacktitle=gray!10, coltitle=green!50!black, attach boxed title to top left={xshift=-2mm,yshift=-2mm}, title={{\Large T}HEOREM}}

\newtcolorbox{question}{skin=enhanced, colback=white, colframe=black, fonttitle=\bfseries, colbacktitle=gray!10, coltitle=green!50!black, attach boxed title to top left={xshift=-2mm,yshift=-2mm}, title={{\Large Q}UESTION}}


\newtcolorbox{warning}{skin=enhanced, colback=gray!10!white, colframe=black, fonttitle=\bfseries, colbacktitle=white, coltitle=red!50!gray, attach boxed title to top left={xshift=3mm,yshift=-2mm}, title={\large Warning!}}

\newtcolorbox{visualconnection}{skin=enhanced, colback=white, colframe=black, fonttitle=\bfseries, colbacktitle=white, coltitle=blue!50!gray, attach boxed title to top left={xshift=3mm,yshift=-2mm}, title={\large\stepcounter{visualconnection} Visual Connection \Alph{visualconnection}}}

\newtcolorbox{remark}{colback=white,colframe=black}



%\usepackage[latin1]{inputenc} %Needed for accented characters?
%\usepackage{amsfonts}
%\usepackage{latexsym}

%To print solutions, use \solutionstrue; To hide solutions, use \solutionsfalse.
%\sol takes two arguments. #1 is the vertical length. #2 is the text.


\newif\ifsolutions
\solutionsfalse

\ifsolutions
    \newcommand{\soln}[2]{\begin{minipage}[c][#1]{\linewidth}{\textcolor{blue}{\textbf{Solution:}}\quad \textcolor{blue}{#2}}\end{minipage}}
    \newcommand{\opsoln}[1]{#1}
    \newcommand{\tblsoln}[1]{\textcolor{blue}{#1}}
\else
    \newcommand{\soln}[2]{\begin{minipage}[c][#1]{\linewidth}{\vfill}\end{minipage}}
    \newcommand{\opsoln}[1]{0}
    \newcommand{\tblsoln}[1]{\textcolor{white}{#1}}
\fi

\ifsolutions
    \newcommand{\sol}[2]{\begin{minipage}[c][#1]{\linewidth}{\textcolor{blue}{}\quad \textcolor{blue}{#2}}\end{minipage}}
    \newcommand{\opsol}[1]{#1}
    \newcommand{\tblsol}[1]{\textcolor{blue}{#1}}
\else
    \newcommand{\sol}[2]{\begin{minipage}[c][#1]{\linewidth}{\vfill}\end{minipage}}
    \newcommand{\opsol}[1]{0}
    \newcommand{\tblsol}[1]{\textcolor{white}{#1}}
\fi


\renewcommand*\contentsname{Table of Contents}


\usepackage{tocloft}
\setlength\cftparskip{7pt}

%From IODE:
\newcommand{\vs}{\vskip.2cm} %customizable command for inserting small vertical space.  Usually appears between paragraphs.
\usepackage[inline,shortlabels]{enumitem} % gives ability to continue with numbering (add [resume] after \begin{enumerate}) AND to make horizontal lists by adding * to enumerate (\begin{enumerate*})

%\newcommand{\ds}{\displaystyle}
\newcommand{\vv}{\vec{v}}
\newcommand{\uu}{\vec{u}}
\newcommand{\yy}{\vec{y}}

\newcommand{\ww}{\textbf{w}}
\newcommand{\xx}{\textbf{x}}
\newcommand{\bb}{\textbf{b}}
\newcommand{\dt}{\frac{d}{dt}}

\newcommand{\RR}{\mathbb{R}}

\newtheorem{thm}{Theorem}
\newtheorem{ex}[thm]{Example}


\theoremstyle{definition}
%\newtheorem{defn}[thm]{Definition}

\begin{document}

%\setcounter{page}{1}
\pagenumbering{arabic}


\begin{center}

{\LARGE \textsc{Chapter 4:  Introduction to Probabilities}}
\end{center}


\noindent  Decision makers, including those in the business world, are greatly influenced by uncertainty.  Probability provides a valuable tool in quantifying uncertainty, leading to greater success in decision-making.

\section*{An Introduction to Probability}

\begin{defn}
A \textbf{probability} is a numerical value between $0$ and $1$ that measures the chance, or likelihood, that a specific event will occur.  
\end{defn}

\noindent Before jumping into some examples, let's discuss some key terms for our study of probability concepts.

\begin{defn}
\begin{itemize}

\item An \textbf{experiment} results in a specific \textbf{outcome} e.g., a coin flip (the experiment) results in a ``heads" or a ``tails" (the outcome).

\item The \textbf{sample space} for an experiment consists of all possible outcomes e.g., $S=\{ \text{heads}, \text{ tails}\}$.  
\begin{itemize}
\item The outcomes must be \textbf{collectively exhaustive} and \textbf{mutually exclusive}.
\begin{itemize}
\item \textbf{Collectively exhaustive} means:  \underline{~~~~~~~~~~~~~~~~~~~~~~~~~~~~~~~~~~~~~~}
\vspace*{.1in}
\item \textbf{Mutually exclusive} means: \underline{~~~~~~~~~~~~~~~~~~~~~~~~~~~~~~~~~~~~~~}
\end{itemize}
\end{itemize}


\item An \textbf{event} is a collection of outcome(s) of an experiment, e.g., a die roll results in ``an odd number".  More specifically, a \textbf{simple event} is an event with a single outcome.

\end{itemize}
\end{defn}

\begin{exercise}
A single die is rolled.
\end{exercise}

\begin{enumerate}[(a)]
\item Define the sample space.
\vfill
\item Define the event, ``an odd number is rolled".  What is the probability of this event?
\vfill

\end{enumerate}

\newpage


\begin{exercise}
A pair of dice are rolled.
\end{exercise}

\begin{enumerate}[(a)]
\item Define the sample space.
\vfill
\item Define the event ``doubles are rolled".  What is the probability of this event?
\vfill
\item Define the event ``the sum of the dice is $10$".  What is the probability of this event?
\vfill

\end{enumerate}



\begin{exercise}
(Donnelly 4.8)  The table below shows the frequencies of executives' salary ranges at a particular organization.
\begin{center}
\begin{tabular}{c c}
\textbf{Salary Range} & \textbf{Frequency}\\ \hline
\text{Under } $\$60,000$ & 2\\
$\$60,000$ \text{ to under } $\$ 70,000$ & 5\\
$\$70,000$ \text{ to under } $\$ 80,000$ & 7\\
$\$80,000$ \text{ to under } $\$ 90,000$ & 4\\
$\$90,000$ \text{ to under } $\$ 100,000$ & 6\\
$\$100,000$ \text{ or more} & ? \\ \hline
\textbf{Total} & $\mathbf{26}$
\end{tabular}
\end{center}
\end{exercise}


\begin{enumerate}[(a)]
\item How many executives at this organization earned $\$100,000$ or more per year?
\vfill
\item What is the probability of randomly selecting an executive who earned $\$100,000$ or more per year?
\vfill
\item What is the probability of randomly selecting an executive who earned $\$60,000$ or more per year?
\vfill
\item What is the probability of randomly selecting an executive  who earned between $\$70,000$ and $\$90,000$ per year?
\vfill
\end{enumerate}



\vfill


\newpage


\begin{statement}
\textbf{Basic Properties of Probability}

\begin{enumerate}

\item If $P(A)=1$, then:
\vspace*{.3in}

\item If $P(A)=0$, then:
\vspace*{.3in}

\item The probability of any event...
\vspace*{.3in}

\item The sum of all probabilities of simple events in a sample space...
\vspace*{.3in}

\item The \textbf{complement}, $A'$, is defined as...
\vspace{.3in}

\begin{itemize}
\item The complement rule:
\vspace*{.3in}
\end{itemize}

\end{enumerate}

\end{statement}



\begin{exercise}
Draw one card from a standard deck of $52$ playing cards.
\end{exercise}
\begin{enumerate}[(a)]
\item What is the sample space for this experiment?
\vfill
\item What is the probability of drawing a ``face card"?
\vfill
\item What is the probability of drawing ``a card divisible by $3$"?
\vfill
\item What is the probability of drawing ``a card NOT divisible by $3$"?
\vfill
\item What is the probability of drawing ``a face card and a card divisible by $3$"?
\vfill
\item What is the probability of drawing ``a red card or a black card"?
\vfill
\end{enumerate}


\newpage

\section*{Probability Rules for More than One Event} 

\begin{defn}
\begin{itemize}

\item A \textbf{contingency table} shows....

\vspace*{.7in}

\item \textbf{Marginal probability} is...

\vspace*{.7in}

\end{itemize}
\end{defn}

\begin{exercise}
(Donnelly 4.19)  A local car dealership currently has $36$ used GM, Ford, and Toyota vehicles on the lot that can be classified as either cars or trucks.  The following data are available:
\begin{itemize}
\item Twenty-six vehicles are cars
\item Eleven vehicles are GMs
\item Fifteen vehicles are Fords
\item Three vehicles are both Toyotas and trucks
\item Fourteen vehicles are both Fords and cars
\end{itemize}
\end{exercise}

\begin{enumerate}[(a)]

\item Create a contingency table that summarizes the data.

\vfill
\vfill
\vfill

\item What is the probability that a randomly selected vehicle is a Toyota?

\vfill

\item Wjhat is the probability that a randomly selected vehicle is a truck?

\vfill

\end{enumerate}

\newpage


\begin{defn}
\begin{itemize}

\item The \textbf{intersection} of Events A and B represents...
\vspace*{.7in}

\underline{Notation:}
\vspace*{.2in}

\item The \textbf{union} of Events A and B represents...
\vspace*{.7in}

\underline{Notation:}
\vspace*{.2in}


\end{itemize}
\end{defn}


\begin{defn}
\begin{itemize}
\item The \textbf{joint probability} of two events is the probability of the intersection of two events.

\item The \textbf{addition rule} for probabilities is used to calculate the probability of...
\vspace*{.5in}

\begin{itemize}
\item It depends on knowing whether or not two events are...
\vspace*{.5in}
\end{itemize}

\begin{enumerate}
\item For mutually exclusive events,
$$ P(A\cup B) = ~~~~~~~~~~~~~ $$
\item For events that are \underline{not} mutually exclusive, 
$$ P(A\cup B)=~~~~~~~~~~~~~~~~~~~~~~~~~~ $$
\end{enumerate}

\end{itemize}
\end{defn}


\begin{exercise} (Continued)
Use the contingency table from the previous exercise to answer the following questions.
\end{exercise}

\begin{enumerate}[(a)]

\item What is the probability that a randomly selected vehicle is either a Ford or a car?

\vfill

\item What is the probability that a randomly selected vehicle is a GM truck?

\vfill

\end{enumerate}


\newpage

\section*{Conditional Probability}

\begin{defn}
\textbf{Conditional probability} is...
\vspace*{1in}

It is calculated using the formula
$$ P(A|B) = ~~~~~~~~~~~~~~~~ $$
\end{defn}


\begin{defn}
\begin{itemize}

\item \textbf{Prior probability} is the probability that an event will occur as determined...
\vspace*{.7in}


\item \textbf{Posterior probability} is a revision of...
\vspace*{.7in}

\end{itemize}
\end{defn}


\begin{exercise} (Continued)
Refer to your contingency table from the previous two exercises to answer the following questions.
\end{exercise}

\begin{enumerate}[(a)]
\item What is the probability that a randomly selected vehicle is a Toyota, given it is a car?

\vfill

\item What is the probability that a randomly selected vehicle is a truck, given it is a Ford?

\vfill

\end{enumerate}


\newpage

\section*{Independent and Dependent Events}

\begin{defn}

\begin{itemize}
\item Two events are \underline{~~~~~~~~~~~~~~~~~~~} if the occurrence of one event has no impact on the occurrence of the other event.  The definition for two independent events is:
$$ \boxed{P(A|B)=~~~~~~~~~~~~~~~~} $$
\item Events are \underline{~~~~~~~~~~~~~~~~~~~} if the occurrence of one event affects the occurrence of another event.
\end{itemize}
\end{defn}

\begin{exercise} (Your Turn 8)
The following table shows the number of adults (in thousands) in the United States who were employed and unemployed in 2018 along with their gender.
\begin{center}
\begin{tabular}{c|c c c}
 & \textbf{Men} & \textbf{Women} & \textbf{Total}\\ \hline
\textbf{Unemployed} & 2,222 & 1,641 & 3,863\\
\textbf{Employed} & 63,510 & 53,255 & 116,765\\
\textbf{Total} & 65,732 & 54,896 & 120,628
\end{tabular}
\end{center}

\vspace*{.2in}


Event A $=$ Person is unemployed

Event B $=$ Person is a man

\vspace*{.2in}

Determine whether Events A and B are independent or dependent.

\end{exercise}

\vfill

\newpage

\begin{statement}
The \textbf{multiplication rule} for probabilities is used to calculate the probability for...
\vspace*{.5in}

It depends on knowing whether or not two events are independent or dependent.
\begin{itemize}
\item For dependent events, $P(A\cap B) = ~~~~~~~~~~~$
\vspace*{.1in}
\item For independent events, $P(A\cap B) = ~~~~~~~~~~~$
\vspace*{.1in}
\end{itemize}
\end{statement}


\begin{exercise}
A company has $140$ employees, of which $30$ are supervisors.  Eighty of the employees are married, and $20\%$ of the married employees are supervisors.  If a company employee is randomly selected, what is the probability that employee is marries and is a supervisor?
\end{exercise}

\vfill
\vfill

\begin{exercise}
According to Statcounter in January 2018, $52\%$ of all web traffic came from mobile phones.  If five people are randomly chosen, what is the probability that all five used mobile phones to access the internet?
\end{exercise}

\vfill



\end{document}
